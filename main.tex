\documentclass[uplatex,dvipdfmx,a4j,10pt]{jsarticle}

% https://gist.github.com/Terminus-IMRC/73bffe6287cd05ec714aed4b4144907d
% http://texdoc.net/texmf-dist/doc/latex/circuitikz/circuitikzmanual.pdf

\usepackage{url}
\usepackage{graphicx}
\usepackage{color}
\usepackage[colorlinks]{hyperref}
\usepackage{amsmath,amssymb}
\usepackage{listings}
\usepackage{subcaption}
\usepackage{fancyhdr}
\usepackage{lastpage}
\usepackage{circuitikz}
\bibliographystyle{tieice}

\fancypagestyle{plain}{%
  \renewcommand{\headrulewidth}{0pt}%
  \fancyhf{}%
  \fancyhead{}%
  \fancyfoot[C]{\thepage\hspace{1pt} / \pageref*{LastPage}}%
}
\pagestyle{plain}

\renewcommand\presectionname{第}
\renewcommand\postsectionname{章}
\renewcommand{\lstlistingname}{リスト}
\renewcommand\descriptionlabel[1]{$\bullet$ \textbf{#1}.}

\title{電流制限回路}
\author{筑波大学天文研究会 45期 \\ 杉﨑行優}
\date{}

\begin{document}
\maketitle
\thispagestyle{plain}
\newpage
\tableofcontents
\listoffigures
\listoftables
\newpage

\section{電流制限回路}

\begin{quote}
  人生は後戻りできない.「運命の1球」「痛恨のミス」というように,
  あの時ああしていれば,こうしていればという後悔をまったくしない人間はいないはずである.
  しかし,幸せな人の多くは,現在までの人生にまず満足している人であろう.
  未来がわからない状況で,
  幸せな人生を送るための行動を選択するということは決してやさしいことではない.
  「フリーターやニートでも今が楽しければいい」という若者が増えてきているが,
  現在楽しいことをしていればトータルで幸せになれるかというと,
  サラ金で借金をして豪遊した後のことを考えてみれば,もちろんNOである.

  人生を送るときに,将来「大きな後悔」をしないで済むようにするのが一つのキーポイントである.
  親は(配属研究室の教授も)勉強しろとうるさい.
  もしかすると後で「あの時遊んでいればもっと楽しかったのに」と思うかもしれないが,
  どのくらいの「もっと」かを考えると,人生のトータルの暗くに比べれば,たいしたことではない.
  人生全体のスコアが100だったとして,高々101か102になっていたかもしれないくらいで,
  逆に勉強をまったくしないでいたらマイナススコアに転落していたかもしれないと思うと,
  これはそれほど腹はたたない.
  もちろん,他に目的があって,
  「代わりに好きな音楽を極めていれば作曲家として大成功していた」というような場合は,
  話は別である.

  しかし,時間の100\%を勉強するというのは通常は逆効果である.
  学生時代に,級友から
  「1日10時間の勉強ペースが維持できなくなった.こんなことでは数学者になれない」と
  深刻な顔で相談され絶句した経験があるが,
  遊びやサークル活動を通じての社交や世間の常識を学ぶのは,
  実は勉強より大切だったりする.極端な話,
  ストレスが溜まって犯罪を起こしたりしたら巨大なマイナススコアになるから,
  8時間自由時間があったら,平均4時間勉強しようかということになる.
  子供の立場でなく親のほうも,「飲み歩く代わりに,子供とすごす時間を増やす」というのは,
  トータルスコアを考えるときっと得になる行動であろう.
  結婚とか就職などという決定は,スコアに大きな影響を及ぼすので,
  十分な考慮が必要であるが,
  たとえば好きな人が別の人のところに逃げてしまっては元も子もないので,
  ぐずぐず考えてもいけない.
  \null\hfill(徳山豪 (2007). ``オンラインアルゴリズムとストリームアルゴリズム'')
\end{quote}

回路素子に過剰な電流を流すと、素子が破壊されるどころか、
発熱や爆発などにより人間にも悪影響が及ぶおそれがある。
このような不可逆的な事態を避けるため、数Wに至る大電力回路では、
素子に流す電流を適切に制限することが必要である。

本章では、抵抗器および半導体素子を用いた電流制限回路を紹介する。
なお、簡単のため、FETとしてはN-ch FET、トランジスタとしてはNPNトランジスタを用い、
ローサイドにおける電流制限回路のみを示した。
どの回路もP-ch FETまたはPNPトランジスタを用いてハイサイドの回路に書き換えられることに
注意していただきたい。


\subsection{抵抗器を用いた電流制限回路}
負荷に直列に抵抗器を接続したとき (図\ref{fig:cur-limit-res})、
負荷に流れる電流の最大値$I_L$は
\begin{equation*}
  I_L = V_{DD} / R_f
\end{equation*}
である。また、抵抗器における最大損失$P_{R_f}$は
\begin{equation*}
  P_{R_f} = I_L R_f
\end{equation*}
である。

\begin{figure}[htb]
  \begin{center}
    \begin{circuitikz} \draw
      (0,0) node[vcc]{$V_{DD}$}
      (0,0) to[twoport,t=負荷,i=$I_L$]
      (0,-2) to[R=$R_f$]
      (0,-4) node[ground]{}
      ;
    \end{circuitikz}
    \caption{抵抗器を用いた電流制限回路}
    \label{fig:cur-limit-res}
  \end{center}
\end{figure}

\begin{itemize}
  \item 利点
  \begin{itemize}
    \item 部品数が少ない。
  \end{itemize}
  \item 欠点
  \begin{itemize}
    \item 抵抗器における損失が死ぬほど大きい。
  \end{itemize}
\end{itemize}


\subsection{FETを用いた電流制限回路}
FETには、$V_{DS}$が十分大きいとき、$I_D$が$V_{GS}$によっておおよそ定まるという性質がある。
例として、2N7000 (N-ch FET) の$I_D - V_{DS}$曲線を図\ref{fig:2n7000-id-vds}に示す。
図より、$V_{GS}=3.0V$のとき最大$I_D=0.1A$、
$V_{GS}=4.0V$のとき最大$I_D=0.4A$となることがわかる。

実際には図\ref{fig:cur-limit-fet}のような回路を用いる。
ここで、$V_{GS}=R_{GS} I_L$であるから$R_f=V_{GS}/I_L$である。

\begin{figure}[htb]
  \begin{subfigure}{.7\textwidth}
    \begin{center}
      \includegraphics[width=.95\linewidth]{2n7000-id-vds.png}
      \caption{2N7000の$I_D - V_{DS}$曲線}
      \label{fig:2n7000-id-vds}
    \end{center}
  \end{subfigure}%
  \begin{subfigure}{.3\textwidth}
    \begin{center}
      \begin{circuitikz} \draw
        (0,0) node[nigfete] (nfet) {}
        (nfet.G) node[anchor=south] {G}
        (nfet.D) node[anchor=east] {D}
        (nfet.S) node[anchor=east] {S}
        (nfet.S) to[open,v=$V_{DS}$] (nfet.D)
        (nfet.G) --
        (-2,-0.25) --
        (-2,-4) --
        (0,-4) to[R=$R_{GS}$,v=$V_{GS}$]
        (nfet.S)
        (0,-4) node[circ] {} |- (0,-5) node[vee]{}
        (nfet.D) --
        (0,2) to[twoport,t=負荷,i<=$I_L$] (0,4) -- (0,4.5) node[vcc]{}
        ;
      \end{circuitikz}
      \caption{FETを用いた電流制限回路}
      \label{fig:cur-limit-fet}
    \end{center}
  \end{subfigure}
  \caption{FETを用いた電流制限回路}
\end{figure}

\begin{itemize}
  \item 利点
  \begin{itemize}
    \item 部品数が少ない。
    \item FETのオン抵抗$R_{DS(on)}$は数$\Omega$から数十m$\Omega$ほどであり、
          FETにおける損失$P_D=R_{DS(on)}I_L^2$が小さい。
    \item FETは負の温度依存性を持つ。すなわち、損失によりFETが発熱しても
          電流値は減少する (安全な) 方向に変化する。
  \end{itemize}
  \item 欠点
  \begin{itemize}
    \item $I_D - V_{DS}$は、特に$V_{DS}$が大きい領域では製品間でかなりバラつきがある。
    \item オン抵抗が小さいFETは$I_D$がかなり大きくないと飽和しないか、
          SOA (Safe Operating Area) の範囲内では飽和しない。
  \end{itemize}
\end{itemize}

\subsection{トランジスタの$V_{BE}$を利用した電流制限回路}
\subsection{センス抵抗とトランジスタを用いた電流制限回路}
\subsection{センス抵抗とオペアンプを用いた電流制限回路}


\clearpage
\bibliography{ref}

\end{document}
